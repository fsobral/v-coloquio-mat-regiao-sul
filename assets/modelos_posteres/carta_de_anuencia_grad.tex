\documentclass[11pt, a4paper, oneside]{article}
\usepackage{amssymb,amsfonts,amssymb,amsthm}
\usepackage[brazil]{babel}
\usepackage{graphicx}
\usepackage{color}
\usepackage[latin1]{inputenc}
\usepackage{helvet}
\renewcommand{\familydefault}{\sfdefault}
\usepackage{setspace} % espacamento entre linhas
\setstretch{1.0}
%%%%%%%%%%%%%%%%%%%%%%%%%%%%%%%%%%%%%%%%%%%%%%%%%%%%%
%
%%%%%%%FORMATA��O DA P�GINA -  POR FAVOR N�O MODIFIQUE!!!!
%
%%%%%%%%%%%%%%%%%%%%%%%%%%%%%%%%%%%%%%%%%%%%%%%%%%%
\pagestyle{myheadings} \topmargin =-20pt \marginparwidth = 45pt
\evensidemargin = -15pt \oddsidemargin = -15pt \textheight = 670pt
\textwidth = 500pt \linespread{1.2}
%%%%%%%%%%%%%%%%%%%%%%%%%%%%%%%%%%%%%%%%%%%%%%%%%%

\begin{document}
\thispagestyle{empty}

\begin{center}
    \textbf{{\Large V Col�quio de Matem�tica da Regi�o Sul}\\
    {\Large 01 a 05 de agosto de 2022}}\\
\end{center}
\begin{center}
    CARTA DE ANU�NCIA\\
\end{center}
\begin{flushright}
    Maring�, xx de julho de 2022.\\
\end{flushright}
\vspace{1cm}

\noindent � Comiss�o Organizadora do V Col�quio de Maten�tica da Regi�o Sul\\
\noindent Universidade Estadual de Maring� - UEM\\
\vspace{3cm}

\noindent Meu aluno, \textbf{Joseph Louis Lagrange}, estudante do curso de Matem�tica, vinculado ao Programa Institucional de Bolsa de Inicia��o Cient�fica \textbf{(PIBIC ou PIC, se for o caso)} pretende apresentar trabalho, na sess�o de p�steres, durante o  V Col�quio de Matem�tica da Regi�o Sul, sob o t�tulo ``Extremos de Fun��es de V�rias Vari�veis Sujeitas � Retri��es'' desenvolvido sob minha orienta��o.\\

\noindent Acredito que este trabalho ser� uma importante contribui��o para o evento e para a forma��o acad�mica do referido aluno. Estou ciente que o trabalho ser� divulgado no site do evento.

\vspace{3cm}



\begin{flushright}
    \begin{minipage}{8cm}
        \begin{center}
            \rule{5cm}{0.5pt}\\
            Leonhard Euler\\
        \end{center}
    \end{minipage}
\end{flushright}







\end{document}
